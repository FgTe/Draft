\documentclass{article}

\usepackage[margin=1cm]{geometry}
\usepackage{amsmath}
\usepackage{fontspec}
\setmainfont{FandolSong-Regular.otf}

\newcommand\D[1]{\Delta{#1}}
\newcommand\F[2]{\frac{#1}{#2}}
\newcommand\di[1]{\text{d}#1}

\begin{document}
  \[\frac{\Delta{Y}}{\Delta{X}} = \frac{aX + bY}{cX + eY}\]
  To solve the given first-order differential equation, let's rewrite it in the standard form:

\[\frac{dy}{dx} = \frac{aX + bY}{cX + eY}.\]

We want to find a function Y(X) that satisfies this equation. First, we can try to determine if this differential equation is separable, homogeneous, or exact.

Separable: A separable differential equation can be written as the product of a function of X and a function of Y, i.e., \(\frac{dy}{dx}=f(X)g(Y)\). In this case, the given equation is not separable, as we cannot separate the variables X and Y.

Homogeneous: A homogeneous differential equation has the form \( \frac{dy}{dx}=f(\frac{Y}{X})\). We can try to find a substitution to make the equation homogeneous. Let's try the substitution \(v=\frac{Y}{X}\), which implies \(Y=vX\). Differentiating Y with respect to X, we get:

\[\frac{dy}{dx}=v + x\frac{dv}{dx}.\]

Now, substitute (Y = vX) into the original equation:

\[\frac{dy}{dx} =v + x\frac{dv}{dx} = \frac{aX + bvX}{cX + evX} = \frac{X(a+bv)}{X(c+ev)}.\]

Canceling the X's and simplifying, we get:
\begin{align*}
  v + x\frac{\di{v}}{\di{x}} &= \frac{a + bv}{c + ev} \\
  \frac{x}{\di{x}} &= \frac{a + bv}{\di{v}(c + ev)} - \frac{v}{\di{v}} \\
  \frac{1}{x}\di{x} &= \frac{c + ev}{a + bv - v(c + ev)} \di{v} \\
  &= \frac{c + ev}{a - cv + v(b - ev)} \di{v}\\
  x\frac{\di{v}}{\di{x}} &= \frac{a + bv - cv - ev^2}{c + ev} \\
  &= \frac{a + bv - cv - ev^2}{c + ev} \\
  &= \frac{a - cv + v(b- ev)}{c + ev} \\
  x\frac{\di{v}}{\di{x}} - \frac{v(b- ev)}{c + ev} &= \frac{a - cv}{c + ev} \\
  x\frac{\di{v}}{\di{x}} \frac{c + ev}{b - ev} - v &= \frac{a - cv}{b - ev} 
\end{align*}
\[\frac{dv}{dx} = \frac{a - cv}{b - ev}.\]

This equation is now separable. Rearrange to separate the variables:

\[\frac{dv}{a-cv} = \frac{dx}{b-ex}.\]

Now, integrate both sides:

\[\int \frac{dv}{a-cv} = \int \frac{dx}{b-ex}.\]

Performing the integration, we obtain:

\[-\frac{1}{c}\ln|a-cv| = -\frac{1}{e}\ln|b-ex| + C,\]

where C is the integration constant. Exponentiate both sides to remove the logarithms:

\[(a-cv)^{-\frac{1}{c}} = (b-ex)^{-\frac{1}{e}}e^{C'}.\]

We can rewrite this as:

\[(a-cv)^{\frac{1}{c}} = (b-ex)^{\frac{1}{e}}e^{-C'},\]

where \(C' = -C\). Now, recall that \(v = \frac{Y}{X}\). Substituting back, we have:

\[\left(a-c\frac{Y}{X}\right)^{\frac{1}{c}} = \left(b-eX\right)^{\frac{1}{e}}e^{-C'}.\]

This is the general solution to the given differential equation. The constant C' can be determined by an initial condition, if provided.
  \begin{gather*}
    y = ax \\
    y = bx \\
    (a\sin{C})^2 + (b-a\cos{C})^2 = c^2 \\
    a^2 - a^2\cos^2{C} + b^2 - 2ab\cos{C} + a^2\cos^2 = c^2 \\
    \cos{C} = \frac{a^2 + b^2 - c^2}{2ab}
  \end{gather*}
  \begin{gather*}
    u = \frac{x}{y} \\
    \begin{gathered}
      u = U(t) \quad x = X(t) \quad y = Y(t)
    \end{gathered} \\
    \begin{aligned}
      U(t) &= \F{Y(t)}{X(t)} \\
      \F{\di{U}(t)}{\di{t}} &= \left( \F{Y(t + \D{t})X(t) - Y(t)X(t + \D{t})}{X(t + \D{t})X(t)} \right) \F{1}{\D{t}} \\
      &= \F{Y'(t)X(t) - Y(t)X'(t)}{X^2(t)} \\
      \di{U}(t) &= \F{\di{Y}(t)X(t) - Y(t)\di{X}(t)}{X^2(t)} \\
      \di{u} & = \F{\di{y}x - \di{x}y}{x^{2}} \\
      \F{\di{u}}{\di{x}} &= \F{\di{y}}{\di{x}}\F{1}{x} - \F{y}{x^2} \\
      x\left( \F{\di{u}}{\di{x}} + \F{y}{x^2} \right) &= \F{\di{y}}{\di{x}} \\
      x\F{\di{u}}{\di{x}} + u &= \F{\di{y}}{\di{x}} \\
      \di{y} &= \di{u}x + u\di{x} \\
    \end{aligned} \\
    \begin{gathered}
      \ln{\F{y}{x}}\di{y} + \F{y}{x}\di{x} = 0 \\
    \end{gathered} \\
    \begin{aligned}
      \ln{u}(\di{u}x + u\di{x}) + u\di{x} &= 0 \\ 
      \ln{u}\di{u}x + ( \ln{u} + 1 ) u\di{x} &= 0 \\
      \ln{u}\di{u} + ( \ln{u} + 1 ) u\F{\di{x}}{x} &= 0 \\
      -\F{\di{x}}{x} = \F{\ln{u}}{\left( \ln{u} + 1 \right)u}\di{u}n
    \end{aligned}
  \end{gather*}
  \begin{gather*}
    ax^{2}+bx+c = 0 \\
    x^{2}+\frac{b}{a}x+\frac{c}{a} = 0 \\
    \begin{aligned}
      z+\frac{c}{a} =& \left(\frac{b}{2a}\right)^{2} \\
      z =& \left(\frac{b}{2a}\right)^{2} - \frac{c}{a} \\
    \end{aligned} \\
    \begin{aligned}
      \left(x + \frac{b}{2a}\right)^{2} - \left( \left(\frac{b}{2a}\right)^{2} - \frac{c}{a} \right) &= 0 \\
      \left(x + \frac{b}{2a}\right)^{2} &= \left( \left(\frac{b}{2a}\right)^{2} - \frac{c}{a} \right) \\
      x &= \pm\frac{\sqrt{b^{2}-4ac}}{2a} - \frac{b}{2a}
    \end{aligned}
  \end{gather*}
  \begin{gather*}
    \left[x\right]^{1}_{0}
    S = 2\pi\frac{a}{2h}(h - x)\Delta{x}\\
    p = v(h - x)\\
    \begin{aligned}
      P = Sp =& \int v(h - x)2\pi\frac{a}{2h}\Delta{x} \\
      =&2v\pi\frac{a}{2h}\int (h - x)^{2} \Delta{x}
    \end{aligned}
  \end{gather*}
  \begin{gather*}
    \theta = \tan{t} \\
    \begin{aligned}
      & \int \sqrt{\theta^{2} + 1} \\
      =& \int \sqrt{\tan^{2}{t} + 1} \sec^{2}{t} \\
      =& \int \sqrt{\sec^{2}{t}} \sec^{2}{t} = \int \sec^{3}{t} \\
      =& \tan{t}\sec{t} - \int \tan{t}\frac{\sin{t}}{\cos^{2}{t}} \\
      =& \tan{t}\sec{t} - \int \tan^{2}{t}\sec{t} \\
      =& \tan{t}\sec{t} - \int (\sec^{2}{t} - 1)\sec{t} \\
      =& \tan{t}\sec{t} - ( \int \sec^{3}{t} - \int \sec{t} ) \\
      =& \frac{\tan{t}\sec{t} + \int \sec{t}}{2} \qquad \frac{1}{\frac{1 + \sin{t}}{\cos{t}}}\frac{\sin{t} + 1}{\cos^{2}{t}} \\
      =& \frac{\tan{t}\sec{t} + \ln{\sec{t} + \tan{t}}}{2}
    \end{aligned}
  \end{gather*}
\end{document}