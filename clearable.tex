\documentclass{article}

\usepackage[margin=1cm]{geometry}
\usepackage{amsmath}
\usepackage{fontspec}
\setmainfont{FandolSong-Regular.otf}

\begin{document}
We can treat $\frac{1}{F'(x)}$ as a constant in the expression:

$$\int \frac{f'(x)}{F'(x)}dx = \frac{1}{F'(x)} \int f'(x) \cdot \frac{1}{F'(x)}du$$

because it does not depend on the variable of integration, which is $u$. Let me explain this in more detail.

When we make the substitution $u = F(x)$, we are essentially changing the variable of integration from $x$ to $u$. So, we need to express the differential $dx$ in terms of $du$ to rewrite the integral in terms of $u$. This is done using the chain rule:

$$dx = \frac{du}{F'(x)}$$

Substituting this into the original integral, we get:

$$\int \frac{f'(x)}{F'(x)}dx = \int \frac{f'(x)}{du} \cdot \frac{dx}{F'(x)}du$$

Simplifying this expression, we get:

$$\int \frac{f'(x)}{F'(x)}dx = \frac{1}{F'(x)} \int f'(x) \cdot \frac{1}{F'(x)}du$$

Note that in this expression, $\frac{1}{F'(x)}$ is a constant with respect to the variable of integration $u$. This is because $u$ is a function of $x$, and $\frac{1}{F'(x)}$ does not depend on $u$. Therefore, we can factor it out of the integral and treat it as a constant:

$$\int \frac{f'(x)}{F'(x)}dx = \frac{1}{F'(x)} \int \frac{df(u)}{du}du = \frac{1}{F'(x)} f(u) + C$$

$$\int \frac{f'(x)}{F'(x)}dx = \frac{1}{F'(x)} f(F(x)) + C$$

So, we can treat $\frac{1}{F'(x)}$ as a constant in this context because it does not depend on the variable of integration $u$.
\end{document}